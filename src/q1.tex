\subsection*{a)}
The law of mass action is stating that the product of hole concentration and free electron concentration is equal to the square of the intrinsic carrier concentration, i.e. $n p = n_i^2$, where $n_i$ is a function of temperature.

Therefore, $n_i$ is constant under thermal equilibrium.
\subsection*{b)}
\[
\begin{aligned}
P(E) &= \frac{1}{1+\exp\left(\frac{E-E_F}{k T}\right)} \\
&\approx \frac{1}{\exp\left(\frac{E-E_F}{k T}\right)} \\
&= \exp\left(\frac{E_F - E}{k T}\right)
\end{aligned}
\]
when $E - E_F \gg k T$, typically $E - E_F > 3 k T $
\subsection*{c)}
$$n = N_C \exp \left(\frac{E_F - E_C}{k T} \right)$$
$N_C$ is the effective density of states, or the maximum number of electrons if every possible state is occupied.
The exponential term is an application of the Fermi function to the conduction band.
\subsection*{d)}
$M_C$ and $M_V$ are, respectively, the conduction band minima and valence band maxima. %\todo[inline]{Explain this}